\documentclass[11pt]{article}
\usepackage{amsmath}
\usepackage{amsfonts}
\usepackage{fancyhdr}
\usepackage{times}
\usepackage{gauss}
\usepackage{tabto}
\usepackage{amssymb}
\usepackage{xfrac}
\usepackage[T1]{fontenc}
\usepackage{geometry}
 \geometry{
 a4paper,
 total={170mm,257mm},
 left=20mm,
 top=20mm,
 }
 \usepackage[usenames,dvipsnames]{xcolor}
 \definecolor{violet}{rgb}{0.5, 0.0, 1.0}
 \usepackage{hyperref}
 \hypersetup{
    colorlinks=true,
    linkcolor=blue,
    filecolor=magenta,      
    urlcolor=violet,
}
\newcommand{\doublerule}[1][.4pt]{%
  \noindent
  \makebox[0pt][l]{\rule[.7ex]{\linewidth}{#1}}%
  \rule[.3ex]{\linewidth}{#1}}
 \newcommand{\bigzero}{\mbox{\normalfont\Large\bfseries 0}}
\newcommand{\rvline}{\hspace*{-\arraycolsep}\vline\hspace*{-\arraycolsep}}
  \linespread{1.2}
\begin{document}
\begin{titlepage}
\begin {center}
\huge {Univerzitet u Nišu\\Elektronski fakultet
\vspace{7.5cm}
\\
\doublerule
\\
\vspace{1cm}
\textbf {\Huge{{Matrični metodi u računarstvu \\ Projekat 1}}}
\vspace{1cm}
\\
\doublerule
\\
\vspace{7.5cm}
\textit{{Stefan Aleksić 16995}\\\Large{02. April, 2020. god.}}}
\end{center}
\end{titlepage}
\newpage
\section*{Projekat br. 1}
Poznata je $LU$ dekompozicija regularne matrice $A=LU \in \mathcal {M}_{n\times n}$. Regularna matrica $M$ ima block formu:\\
\begin{equation*}
M=
\begin{bmatrix}
A & v \\ u^T & a
\end{bmatrix},
\end{equation*}
\\gde su $v, u\in \mathbb{R}^n $ vektori i $ a \in \mathbb{R} $ nenula skalar. Odrediti LU faktorizaciju matrice M u blok formi\\
\begin{equation*}\tag{1}\label{eq1}
M=
\begin{bmatrix}
\ L\  & \_\_\ \\ \_\_ & \_\_\ 
\end{bmatrix}
\cdot
\begin{bmatrix}
\ U\  & \_\_\ \\ \_\_ & \_\_\ 
\end{bmatrix},\quad (A=LU)
\end{equation*}
i odrediti potreban broj aritmetičkih operacija za njeno dobijanje. U izrazu \eqref{eq1} pozicije \_\_ označavaju vektore i skalare koje je potrebno odrediti tako da matrica
$ 
\begin{bmatrix}
\ L\  & \_\_\ \\ \_\_ & \_\_\ 
\end{bmatrix}
$
bude donje trougaona, matrica
$
\begin{bmatrix}
\ U\  & \_\_\ \\ \_\_ & \_\_\ 
\end{bmatrix}
$
bude donje trougaona i važi jednakost \eqref{eq1}.\\
Kako se menja broj potrebnih aritmetičkih operacija u slučaju kada su $A$ i $M$ simetrične matrice?
\subsection*{Rešenje:}
Prvo ćemo označiti elemente blok matrica $L_M$ i $U_M$.
\begin{equation*}
M=L_MU_M=
\begin{bmatrix}
L & \overrightarrow{l_{12}} \\ \overrightarrow{l_{21}}^T & l_a
\end{bmatrix}
\cdot
\begin{bmatrix}
U& \overrightarrow{u_{12}} \\ \overrightarrow{u_{21}}^T & u_a
\end{bmatrix},\quad
\overrightarrow{l_{12}}, \overrightarrow{l_{21}}, \overrightarrow{u_{12}}, \overrightarrow{u_{21}} \in \mathbb{R}^{n},\quad l_a, u_a \in \mathbb{R}
\end{equation*}
S obzirom da su matrice $L_M,U_M$ donje i gornja trogaona respektivno, odavde možemo da zaključimo da je $\overrightarrow{l_{12}}= \overrightarrow{u_{21}}=\overrightarrow{0}$. Hajmo sada da pomnožimo matrice $L_M$ i $U_M$:
\begin{equation*}
L_MU_M=
\begin{bmatrix}
L & \overrightarrow{0} \\ \overrightarrow{l_{21}}^T & l_a
\end{bmatrix}
\cdot
\begin{bmatrix}
U& \overrightarrow{u_{12}} \\ \overrightarrow{0}^T & u_a
\end{bmatrix}
=
\begin{bmatrix}
LU  & L\overrightarrow{u_{12}} \\
\overrightarrow{l_{21}}^TU & \overrightarrow{l_{21}}^T\overrightarrow{u_{12}} + l_au_a
\end{bmatrix}
=
\begin{bmatrix}
A & \overrightarrow{v} \\ \overrightarrow{u}^T & a
\end{bmatrix}
=M
\end{equation*}
Izjednačavanjem leve i desne matrice dobijamo sistem:
\begin{equation*}\tag{3}\label{eq3}
\iff 
\begin{cases}
LU=A\\
L\overrightarrow{u_{12}} = v \implies \overrightarrow{u_{12}}=L^{-1}v \\
\overrightarrow{l_{21}}^TU = u^T \implies \overrightarrow{l_{21}}^T=u^TU^{-1}\\
\overrightarrow{l_{21}}^T\overrightarrow{u_{12}} + l_au_a=a \implies u^T(U^{-1}L^{-1})v + l_au_a=u^T(LU)^{-1}v + l_au_a=u^TA^{-1}v + l_au_a=a
\end{cases}
\hspace{5cm}
\end{equation*}
Iako smo gornjim sistemom opisali skup matrica $(l_a,u_a \in \mathbb{R})$, mi možemo na svoju ruku izabrati ono rešenje koje je pogodno za računicu, odnosno ono koje bi bilo najzgodnije po konvenciji. S obzirom da se na glavnoj dijagonali matrice $L_m$ očekuju jedinice, odnosno
\begin{equation*} det(L_m)=det(L)det(l_a)=1 \implies l_a=1 \land u_a=a-u^TA^{-1}v=M/A
\end{equation*}
Odavde konačno dobijamo elemente matrica $L_M$ i $U_M$:
\begin{equation*}\tag{4}\label{eq4}
L_M=
\begin{bmatrix}
L & \overrightarrow{0}\ \\
u^TU^{-1} & 1
\end{bmatrix}, \quad
U_M=
\begin{bmatrix}
U & L^{-1}v\\
\overrightarrow{0} & M/A
\end{bmatrix}
\end{equation*}
Aritmetičke operacije:
\begin{center}
\begin{tabular}{ |l || c | c | c || c | }\hline
 Oznaka operacije & Sabiranje/Oduzimanje & Množenje & Deljenje & \textbf{Ukupno} \\ \hline \hline
 $ \overrightarrow{l_{21}}^TU=u^T $ & $ \frac{\displaystyle{n(n-1)}}{\displaystyle{2}} $ &  $ \frac{\displaystyle{n(n-1)}}{\displaystyle{2}}  $ & $ n $ & $ \boldsymbol{\mathcal{O}n ^2} $\\ \hline
 $ L\overrightarrow{u_{12}}=v $ & $ \frac{\displaystyle{n(n-1)}}{\displaystyle{2}} $ & $ \frac{\displaystyle{n(n-1)}}{\displaystyle{2}} $ & $ 0 $ & $ \boldsymbol{\mathcal{O}n^2} $ \\ \hline
 $ M/A=a-\overrightarrow{l_{21}}^T\overrightarrow{u_{12}} $ & $ n $ & $ n $ & $ 0 $ & $ \boldsymbol{\mathcal{O}n} $ \\ \hline \hline
 Ceo algoritam & $ n^2 $ & $ n^2 $ & $ n $ & $ \boldsymbol{\mathcal{O} 2n^2} $ \\ \hline
\end{tabular}
\end{center}
\subsubsection*{Napomene:} *Izraze: $u^TU^{-1} $ i $L^{-1}v$ izjednačavamo vektorima $\overrightarrow{l_{21}}^T$ i $\overrightarrow{u_{12}} $ respektivno, a onda izraze računamo kao sisteme jednačina:$$\overrightarrow{l_{21}}^TU=u^T \quad \text{i} \quad L\overrightarrow{u_{12}}=v$$
*S obzirom da matrica $L$ na glavnoj dijagonali ima jedinice, za sistem $L\overrightarrow{u_{12}}=v$ nije potrebno deljenje.\\
*Šurov komplement $M/A$ odavde je: $$M/A=a-u^TA^{-1}v=a-u^T(LU)^{-1}v=a-(u^TU^{-1})(L^{-1}v)=a-\overrightarrow{l_{21}}^T\overrightarrow{u_{12}}$$
 *Sve što smo jednom izračunali smatramo pribeleženim, kako se ne bi nagomilavala računanja.\vspace{5mm}\\
Ukoliko je matrica $M$, onda važi:
$$ M^T=M \implies
M^T
=
\begin{bmatrix}
A & v \\ u^T & a
\end{bmatrix}^T
=
\begin{bmatrix}
A^T & (u^T)^T \\ v^T & a
\end{bmatrix}
=
\begin{bmatrix}
A^T & u \\ v^T & a
\end{bmatrix}
=
\begin{bmatrix}
A & v \\ u^T & a
\end{bmatrix}
=
M$$
\begin{equation*}\tag{5}\label{eq5}
\iff 
\begin{cases}
A=A^T \\
u=v \\
v^T = u^T \\
a=a
\end{cases}
\implies 
\begin{cases}
A=A^T \\
v=u
\end{cases}\vspace{3mm}
\end{equation*}
Za simetričnu matricu $A$ važi:\\
\begin{equation*}\tag{5.1}\label{eq5.1}
A^T=A \implies (LDU')^T=(DU')^TL^T=U'^TD^TL^T=U'^T D L^T=LDU'
$$$$
\implies L^T=U' \iff L=U'^T
\end{equation*}
Na osnovu  \eqref{eq3}, \eqref{eq5} i \eqref{eq5.1}:
$$L\overrightarrow{u_{12}}=v=u=U^T\overrightarrow{l_{21}} \implies L\overrightarrow{u_{12}}=(DU')^T\overrightarrow{l_{21}}=U'^TD\overrightarrow{l_{21}}\implies \overrightarrow{u_{12}}=L^{-1}LD\overrightarrow{l_{21}}\implies \overrightarrow{u_{12}}=D\overrightarrow{l_{21}}$$
S obzirom da je matrica $D$ dijagonalna matrica, čiji su elementi jednaki elementima glavne dijagonale matrice $U$, za njeno dobijanje nije potrebno nikakvo računanje, dok je za izraz $\overrightarrow{u_{12}}=D\overrightarrow{l_{21}}$ potrebno izvršiti samo $n$ množenja ( $u_{12_i}=d_{ii} \cdot l_{21_i}$  ,  ($ u_{12_i}\in \overrightarrow{u_{12}}$, $d_{ii} \in D_{n \times n}$, $ l_{21_i} \in \overrightarrow{l_{21}}$ $ \forall i=1,n$ )).\vspace{5mm}\\
Aritmetičke operacije za simetričnu matricu $M$:
\begin{center}
\begin{tabular}{ |l || c | c | c || c | }\hline
 Oznaka operacije & Sabiranje/Oduzimanje & Množenje & Deljenje & \textbf{Ukupno} \\ \hline \hline
$ \overrightarrow{l_{21}}^TU=u^T $ & $ \frac{\displaystyle{n(n-1)}}{\displaystyle{2}} $ &  $ \frac{\displaystyle{n(n-1)}}{\displaystyle{2}}  $ & $ n $ & $ \boldsymbol{\mathcal{O}n ^2} $\\ \hline
 $ \overrightarrow{u_{12}}=D\overrightarrow{l_{21}} $ & $ 0 $ & $ n $ & $ 0 $ & $ \boldsymbol{\mathcal{O}n^2} $ \\ \hline
 $ M/A=a-\overrightarrow{l_{21}}^T\overrightarrow{u_{12}} $ & $ n $ & $ n $ & $ 0 $ & $ \boldsymbol{\mathcal{O}n} $ \\ \hline \hline
 Ceo algoritam & $ \frac{\displaystyle{n(n+1)}}{\displaystyle{2}} $ & $ \frac{\displaystyle{n(n+3)}}{\displaystyle{2}} $ & $ n $ & $ \boldsymbol{\mathcal{O} n^2} $ \\ \hline
\end{tabular}
\end{center}
\end{document}